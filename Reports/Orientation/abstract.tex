\begin{abstract}
In this report, the different tools that could play a role for the Mobile Video-on-Demand project are elucidated. The main research question is: \textit{``How can we make video-on-demand available for mobile devices using a non-centralized approach?''}. A number of sub questions are derived from the main research question which deal with the different aspects of the main question separately. Android will be used as mobile platform because it holds the greatest market share and the team is proficient in programming for this platform. A number of Video-on-Demand solutions are described. In the end, Tribler is chosen because it uses a non-centralized approach, is free of costs for the user, has a large amount of videos available in HD and shows no advertisements. Tribler is not yet available for Android so it will have to be ported by using a python for Android tool. A multimedia framework must also be in the prototype for the playback of videos, VLC for Android Beta is chosen due to its state of maturity, popularity, number of codecs and modularity. If VLC should not work when the team attempts to implement it, Stagefright will be used as multimedia framework. If Python for Android does not work, Tribler can not be used and a plug-in will be written for VLC to incorporate main functionality. 
\end{abstract}
