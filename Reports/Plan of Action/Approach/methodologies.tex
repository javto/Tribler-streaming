\section{Methodologies}
\subsection{Scrum}
During the project several methodologies will be put into practice. One of these methodologies, namely Scrum\footnote{http://www.scrum.org}, will be used in all phases starting from the design phase. Scrum is an iterative and incremental Agile method. Scrum uses sprints in which the team goes through the process of adding functionality to the software, always keeping a working product. We chose to do sprints of one week to quickly see the progression and to be able to keep steering things in the right direction. Traditionally, Scrum makes use of a number of roles to assign the different responsibilities to different members of the team. Since our team consists out of two persons we chose not to assign these roles and have all responsibilities shared. The weekly Scrum meetings, the sprint plannings, will be attended by our supervisor. In this manner, he has a better insight and  is able to provide more meaningful feedback. In the sprint planning the following is discussed:
\begin{itemize}
\item[-]What was completed during the last sprint (demo it, if possible).
\item[-]How did it go; what went well and what we can improve. 
\item[-]Select what work is to be done in the upcoming sprint
\item[-]Determine the time it will take to do that work and divide it amongst the team.
\end{itemize}
The daily scrum meetings will only be attended by the team and they will discuss what each person did yesterday, what each person is going to do and if there are any impediments that are needed to overcome. Furthermore, demo sessions for the client will be held at his request and also on a two-weekly basis to keep him informed about the progress we have been making.
\subsection{MoSCoW}
MoSCoW is a de facto standard in prioritizing a list of requirements into the following categories:
\begin{itemize}
\item[-] Must have: requirements that must be satisfied in the final solution for the solution to be considered a success.
\item[-] Should have: high-priority requirements that should be included in the solution if possible.
\item[-] Could have: requirements which are considered desirable but not necessary.
\item[-] Would have: requirements that will not be implemented in a given release, but may be considered for the future.\\
\\
The MoSCoW method will be used to prioritize the elicited requirements in the Requirements Analysis Document.
\subsection{Test Driven Development}
Test driven development (TDD) is a methodology in which , rather than the . This approach is used to get a better understanding of the system.  function is of what you’re writing the test for. 
