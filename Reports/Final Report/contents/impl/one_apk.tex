\chapter{Combining VLC and Libtorrent}
\thispagestyle{fancy}
\label{sec:one_apk}
When using Android, one of the main annoyances when downloading a new application is that sometimes the user must first install a separate application for some functionality. The Team wanted to circumvent this by delivering the streaming application with VLC built-in. This way, the user can search a video, play and not have to worry about different media frameworks or otherwise. To achieve this, the Team tried to let VLC act like a library from which the libtorrent client application could call functions for playback. However, VLC for Android wasn't build with this purpose in mind, as the team discovered after some experimentation. The Team fused the two projects together by putting the source of the libtorrent client application together with VLC in one project. The Team now had one .apk file which could be installed, which first downloaded the file and then give the assignment to the VLC part of the prototype to play the media file. 